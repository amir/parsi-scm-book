\chapter{پیشگفتار}
عالیه! شما واقعا دارید این متن رو می‌خونید. این شمارو در یکی از این ۳ دسته قرار میده:
\begin{enumerate}
\item یک دانشجو/برنامه‌نویس/مدیرسیستم که مجبورش کردند استفاده از نرم‌افزارهای مدیریت کدمنبع رو یاد بگیره،
\item یکی که به صورت کاملا اتفاقی این مستند رو دریافت کرده،‌ و از روی کنجکاوی داره نگاهی بهش میندازه،
\item یکی که واقعا به این مفاهیم علاقه‌منده و می‌خواد عمیق‌تر با این مفاهیم آشنا بشه.
\end{enumerate}

شاید این سؤال برای شما مطرح باشه که اصلا چرا باید از نرم‌افزارهای مدیریت کد‌منبع\Footnote{Source Code Management (SCM)} یا در حالت کلی‌تر سامانه‌های مدیریت نسخ\Footnote{Version Control System (VCS)} استفاده کرد؟

بیایید با یک مثال ساده اهمیت موضوع رو بررسی کنیم:

اگر من به شما بگم که من یک ویرایشگر متن فوق‌العاده و یا حتی یک IDE عجیب و غریب پیدا کردم که همه کار می‌کنه و کاملا هم رایگان هست، فقط یک مشکل کوچیک داره و اون اینه که قابلیت "Undo" نداره، واکنش شما چه خواهد بود؟ آیا جرأت استفاده از اون رو در پروژه‌های جدی و بزرگتون دارید؟\footnote{ نه، ندارید!}

همه ما برنامه‌نویسها\footnote{هرچند که در این کتاب گاها به مسائل و مشکلات مربوط به برنامه‌نویس‌ها اشاره میشه، اما مفاهیم مطرح شده، در مورد هرنوع مستند قابل رهگیری صدق می‌کنند.} اشتباه می‌کنیم، و یا نه، تصمیم به انجام تغییرات رادیکالی می‌گیریم و یا حتی نه، یک نرم‌افزار را به مرحله قابل‌قبولی رسانده، آنرا منتشر کرده، مشغول اضافه کردن امکانات جدید برای نسخه‌های بعدی می‌شویم و ....

در تمامی این حالات یک دل‌مشغولی همواره وجود دارد و آن این است که چگونه می‌توان کدهای برنامه را در تمامی این حالتها نگه‌داری کرد و در صورت لزوم و به سهولت به وضعیت یک فایل در مثلا ۶ ماه و ۲۳ روز پیش برگشت؟ یا مثلا چجوری میشه یک شاخه از کدهای اصلی برنامه جدا کرد، دیوانه‌وار آنرا تغییر داد و در صورت رضایت‌بخش بودن نتیجه آنرا با کد اصلی برنامه یکی کرد؟ یا اینکه چطور میشه همزمان با چندین برنامه‌نوس دیگر (و نه در یک اتاق یا پارتیشن، بلکه در طرف دیگر کره زمین) به طور مشترک روی یک پروژه کار کرد بدون اینکه مشکلی پیش بیاد؟

اگر شما نیز با چنین مشکلاتی مواجه هستید و یا فکر می‌کنید که مواجه خواهید شد، این کتاب مناسب شماست چون سعی می‌کند که راه‌حلهای مناسبی را برای این مشکلات با استفاده از نرم‌افزارهای بازمتن و کاملا رایگان آموزش دهد.

این کتاب با استفاده از \mbox{\lr{\TeX\\}} و \mbox{\lr{\XeTeX\\}} نوشته شده است، توزیع تکلایو ۲۰۰۸ \footnote{قابل دریافت از: http://tug.org/texlive/}تمامی نیازهای نوشتن این کتاب (از جمله بسته \mbox{\lr{\XePersian\\}} که برای پارسی نویسی استفاده شده\footnote{با تشکر ویژه از گروه فارسی-لاتک،‌ آقایان: مصطفی واحدی، وفا خلیقی و مهدی امیدعلی}) را برآورده کرده.\\
\paragraph{}
{\sols فعلا عمدا سفید گذاشته شده!}
